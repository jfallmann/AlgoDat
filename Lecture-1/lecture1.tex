%% ===================================================================
%%% RCS Header:
%%% $RCSfile: wuerzburg-jun16.tex,v $
%%% $Revision: 1.4 $
%%% $Date: 2016/06/10 09:40:15 $
%%% $Author: backofen $
%%% $Locker:  $
%%% ===================================================================
\NeedsTeXFormat{LaTeX2e}
\listfiles
\setcounter{errorcontextlines}{\maxdimen}

\newcommand{\percent}{\%}
\documentclass[notes=hide,pdftex,14pt]{beamer}
\usepackage[utf8]{inputenc}  
\usepackage[german,english]{babel}                                          

\usepackage{tabitem-beamer}
\usepackage{align-objs}
\usepackage{basiccolors}
\definecolor{Mittel-Gruen}{rgb}{0,.933,0}
\definecolor{Mittel-Blau}{rgb}{0,0,.933}
\definecolor{Hell-Blau}{rgb}{0,0,.633}
\newcommand{\currentred}{red!70!gray}
\newcommand{\currentgreen}{red!70!gray}
%%% Math extensions                
%%%                            
\usepackage{latexsym}              
\usepackage{amssymb}              
\usepackage{amsmath}           
\usepackage{array}             
\usepackage{delarray}             
\usepackage{tabularx}             


%%% Packages
\usepackage[binary-units=true]{siunitx}

 \usepackage{adjustbox}
% \usepackage{capt-of}

% \usepackage{tabularx, colortbl} % Tables
 \usepackage{booktabs}

 \usepackage{tikz}
 \usetikzlibrary{patterns,arrows,decorations.pathreplacing,shapes.geometric,calc}

%\usepackage{wasysym} % Checkmark etc.

\usepackage{listings} % Code
\input{Styles/Code/listings-python.prf} % Python style
\input{Styles/Code/listings-bash.prf} % Bash style

\usepackage{qrcode}


\usetheme[CID]{Freiburg}
\usepackage{beamerseminar}

\usepackage[autostyle=true]{csquotes}
\usepackage{ifthen}

% We need topics for our sources for grouping
% old: \usepackage[backend=bibtex]{biblatex}
\usepackage[defaultbib]{bibtopic}
\bibliographystyle{alpha}
\renewcommand{\thebtauxfile}{\jobname.\arabic{btauxfile}.btaux}

%%% Source of eatdot
%%%http://tex.stackexchange.com/questions/152892/how-to-delete-a-full-stop-on-reference-ending
\newcommand\eatdot[1]{}

%%% Source of subtoc-fix
%%% 
%%%http://stackoverflow.com/questions/2795478/latex-beamer-prevent-showing-the-toc-at-one-occation
\newboolean{sectiontoc}
\setboolean{sectiontoc}{true} % default to true
\newcommand{\toclesssection}[1]{
  \setboolean{sectiontoc}{false}
  \section{#1}
  \setboolean{sectiontoc}{true}
}

\def\LectureProgLanguage{python}
\def\LectureDesign{plain}
\def\LectureToC{Overview}
\def\PresentationLectureNumber{1}
\def\PresentationExerciseFeedback{false}% Include the feedback of the exercises?
\def\PresentationLectureFeedback{false}% Include the feedback of the lecture?

\def\PresentationAuthor{Prof.\ Dr.\ Rolf\ Backofen}
\def\PresentationAuthorText{Prof.~Dr.~Rolf~Backofen}

%-------------------------------------------------------------------------------
% Special lecture-specific definitions
%-------------------------------------------------------------------------------
\def\PresentationDate{March~2016}

\def\LectureHomepageLink
  {http://www.bioinf.uni-freiburg.de/Lehre/Courses/2016_WS/V_AuD/}
\def\LectureDaphneLink
  {https://daphne.informatik.uni-freiburg.de/}
\def\LectureForumLink
  {https://daphne.informatik.uni-freiburg.de/forum/}
%\def\LectureCheckstyleLink
%  {https://pypi.python.org/pypi/pep8}
\def\LectureCheckstyleLink
  {https://pypi.python.org/pypi/flake8}
%\def\LectureGitLink
%  {https://git-scm.com/}
\def\LectureSubversionLink
  {https://subversion.apache.org/}
\def\LectureJenkinsLink
  {https://daphne.informatik.uni-freiburg.de/jenkins/}
\def\PresentationTitle{Algorithms and Data Structures}
\def\PresentationDescription{Runtime analysis Introduction/Minsort/Heapsort}
\def\PresentationSmallTitle{\PresentationTitle}% Will be displayed on the lower left of the first slide

\def\PresentationInstitute{BioInf}
\def\PresentationInstituteText{Bioinformatics Group / Department of Computer Science}

\def\LectureToC{Structure}
\def\LectureFeedbackSection{Feedback}

\def\LectureFeedbackExercisesSection{Exercises}
\def\LectureFeedbackExercisesTitle{Feedback from the exercises}

\def\LectureFeedbackLectureSection{Lecture}
\def\LectureFeedbackLectureTitle{Feedback from the lecture}

\def\LectureFurtherLiterature{Further Literature}

%\ufcdSetup{mainlanguage=english}

%-------------------------------------------------------------------------------
% Special lecture-specific definitions
%-------------------------------------------------------------------------------

\def\LectureHomepageLabel
{Homepage}

\def\LectureOrganisationLecture
{Tuesday, 12:15- 13:45, SR 00 010/014, Build. 101}
\def\LectureOrganisationHelp
{Wednesday, 12:15-13:00 - SR 00 010/014, Build. 101 }
\def\LectureOrganisationExam
{Planned: Sa. 25. März 2017, 10:00-12:00 Uhr, Build. 101, Hörsäle 026 und 036}
\def\LectureOrganisationTutors
{{\color{Mittel-Blau}Tobias Faller}, {\color{Mittel-Blau}Till Steinmann}
  and {\color{Mittel-Blau}Tim Maffenbeier}}
\def\LectureOrganisationAdditional
  {{\color{Mittel-Blau}Michael Uhl, Florian Eggenhofer} and {\color{Mittel-Blau} Björn Grüning}}

\def\LectureOrganisationExercisesWorkingTime
{ESE: {\color{Mittel-Blau}1 week}, IEMS: {\color{Mittel-Blau}3 weeks (optional)}}
%-------------------------------------------------------------------------------
\newboolean{subsectiontoc}
\setboolean{subsectiontoc}{true} % default to true
\newcommand{\toclesssubsection}[1]{
  \setboolean{subsectiontoc}{false}
  \subsection{#1}
  \setboolean{subsectiontoc}{true}
}

\ifthenelse{\equal{\LectureProgLanguage}{all}}{%
  \newcommand{\codeslide}[2]{#2}%
}{%
  \newcommand{\codeslide}[2]{%
    \ifthenelse{\equal{#1}{\LectureProgLanguage}}{#2}{}%
  }%
}%

\AtBeginSection[]
{
  \ifthenelse{\boolean{sectiontoc}}{
    \begin{frame}<beamer>{\LectureToC}
      \tableofcontents[currentsection]
    \end{frame}
  }
}

 \AtBeginSubsection[]
 {
   \ifthenelse{\boolean{subsectiontoc}}{
     \begin{frame}<beamer>{\LectureToC}
       \tableofcontents[currentsection,currentsubsection]
     \end{frame}
   }
 }

\title{\PresentationTitle\\
WS 2016 / 2017\\
\PresentationDescription}
\author{Prof. Dr. Rolf Backofen \\
  Lehrstuhl für Bioinformatik \\
 Institut für Informatik \\
Universität Freiburg
}
\date{Lecture 1, Tuesday, 25. Oktober 2016\\
}
\begin{document}
\begin{frame}
  \titlepage
\end{frame}

\toclesssection{Algorithms and Data Structures}
\begin{frame}{Topics of the Lecture}
  \begin{tabl}
    \item Algorithms and Data Structures
    \eitem    \begin{tabp}[0.9]<2->     \item  $\ldots$ for problems that occur in practical \textbf{any}
      larger program/project
\sitem{}
    \item
      \textbf{Algorithm} $=$ Solving of complex computional problems
\sitem{}
    \item
      \textbf{Data Structure:} $=$ Representation of data on computer
    \sitem
  \citem \grah{0.5}{Rolfs-Images/cartoon-recommendation-python-machine-learning.jpg}
      \end{tabp}
  \end{tabl}
\end{frame}


%-------------------------------------------------------------------------------

\begin{frame}[t]{Example 1: Sorting}
  \hfill
%  \qrcode[height=6em]{https://www.youtube.com/watch?v=kPRA0W1kECg}\\
  \vspace*{-1.5em}
  \begin{figure}
    \begin{center}
      \href{https://www.youtube.com/watch?v=kPRA0W1kECg}{
        \begin{adjustbox}{width=0.75\linewidth}%
          \graw{0.7}{Rolfs-Images/sorting.png}
        \end{adjustbox}
      }
    \end{center}
    \caption{%
      \href{https://www.youtube.com/watch?v=kPRA0W1kECg}{%
        Sorting with \textit{Minsort}
      }%
    }%
    \label{fig:minsort}%
  \end{figure}
  \begin{center}
    \url{https://www.youtube.com/watch?v=kPRA0W1kECg}%
  \end{center}
\end{frame}


%-------------------------------------------------------------------------------

\begin{frame}{Example 2: Navigation}
  \begin{tabcl}
  \eitem 
  \iitem{-2em} 
\includegraphicscenter[width=0.65\textwidth]{Rolfs-Images/route.png}
        \sitem 
        \eitem \copyright\,%
          \href{http://openstreetmap.org/}{OpenStreetMap}%
        %
  \end{tabcl}%
\rlap{
\begin{tabcp}[0.35]
        \item<2->
          \textbf{Data Structures:}
        \eitem How to represent the map as data?
        \eitem 
        \item<3->
          \textbf{Algorithms:}
        \eitem How to find the shortest / fastest way?
  \end{tabcp}}
  \end{frame}


\begin{frame}
  \frametitle{Example 3: Fault Tolerant Search}
  \begin{tabl}
  \iitem{-2em} \graw{1.1}{Rolfs-Images/edit-distance.png}
  \end{tabl}
\end{frame}

\begin{frame}
  \frametitle{Example 4: Search for Proteins}
  \begin{tabp}
  \item \textbf{Algorithm for Edit Distance defined new paradigma}
  \eitem \onslide<2-> \graw{1}{Rolfs-Images/blast.png}
\eitem \onslide<3-> \graw{1}{Rolfs-Images/blast2.png}
  \end{tabp}
\end{frame}

\begin{frame}
  \frametitle{Moore's Law and Sequencing}
  \begin{tabl}
  \iitem{1em}\graw{0.9}{Rolfs-Images/science-big-bang.png}
\item<2-> sequencing data exponential outgrows Moore's law 
  \end{tabl}
\end{frame}


\subsection{Content}




\begin{frame}{Content of the Lecture 1 / 2}
  \textbf{General:}
  \begin{itemize}
    \item Most of you had a lecture on basic progamming
      \begin{center}
        \textit{performance was not an issue}
      \end{center}
\vspace*{1em}
    \item<2->
      \textbf{Here} we will tackle the following questions:
      \begin{enumerate}
        \item<3->
          How fast is our program?
        \item<4->
          How can I make it faster?
        \item<5->
          How can I proof that it will always be that fast?
      \end{enumerate}
\vspace*{1em}    \item<6->
      \textbf{Important} issues:
      \begin{itemize}
        \item
          most of the time: \textbf{runtime} of the program
        \item sometimes also:  consumption of resource (\textbf{space} $\ldots$)
      \end{itemize}
  \end{itemize}
\end{frame}

%-------------------------------------------------------------------------------

\begin{frame}{Content of the Lecture 2 / 2}
  \begin{tabl}
  \iitem{-2em}  \textbf{Algorithms:}
\eitem 
      \begin{tabl}<2->
        \item
          Sorting
        \sitem 
        \item
          Dynamic Arrays
        \sitem 
        \item
          Associative Arrays
        \sitem 
        \item
          Hashing
      \end{tabl}
      \begin{tabl}<2->
        \item
          Priority Queue
        \sitem 
        \item
          Linked Lists
        \sitem 
        \item
          Pathfinding / Dijkstra Algorithm
        \sitem 
        \item
          Search-Trees
      \end{tabl}
        \eitem 
    \iitem{-2em}  \onslide<3->\textbf{Mathematics:}
  \eitem 
      \begin{tabl}<4->
        \item
          Runtime analysis
        \sitem 
        \item
          $\mathcal{O}$-Notation
        \sitem 
    \item<5->  (some) proof of correctness
        \sitem 
    \item<6-> overall: very basic mathematics
    \end{tabl}
    \end{tabl}
\end{frame}

\begin{frame}
  \frametitle{After the lecture \ldots{}}
  \begin{tabl}
  \item \ldots{} should you be able to understand the joke
  \citem\graw{0.65}{Rolfs-Images/tree.png}
\citem \url{http://xkcd.com/835/}
\item<2-> hopefully your parents will still invite you 
  \end{tabl}
\end{frame}



\subsection{Links}

\begin{frame}[t]{Links}

  \hfill \qrcode[height=6em]{\LectureHomepageLink}\\
  \vspace*{-0.5em}
  \textbf{Homepage:}
  \begin{itemize}
    \item
      Exercise sheets
    \item
      Lectures
    \item
      Materials
  \end{itemize}
  \begin{center}
    Link to
    \color{Bluea}\href{http://www.bioinf.uni-freiburg.de/Lehre/Courses/2015_WS/V_AuD}{\url{\LectureHomepageLink}}
  \end{center}
\end{frame}
\subsection{Organisation}

\begin{frame}{Organisation 1 / 5}
  \begin{tabp}[1.05]
  \iitem{-2em}  \textbf{Lecture:}
    \item
      {\color{Greenb}\LectureOrganisationLecture}
    \item
      Recordings of the lecture will be uploaded to the 
    \eitem webpage
    \sitem 
\iitem{-2em}  \textbf{Exercises:}
    \item
      One exercise sheet per week
    \item
      Submission / Correction / Assistance online
    \item
      Tutorial: (if needed)
    \sitem 
      {\color{Greenb}\LectureOrganisationHelp}
    \sitem 
\iitem{-2em}  
  \textbf{Exam:}
    \item
      {\color{Greenb}\LectureOrganisationExam}
  \end{tabp}
\end{frame}

%-------------------------------------------------------------------------------

\begin{frame}{Organisation 2 / 5}
  \textbf{Exercises:}
  \begin{itemize}[<+->]
    \item
      {\color{Mittel-Blau}$\SI{80}{\percent}$} practical,     {\color{Mittel-Blau}$\SI{20}{\percent}$} theorethical
    \item we expect \textbf{everyone} to solve \textbf{every} exercise sheet
    \item  \textcolor{DarkGreen}{$\ldots$ but we will not force anyone}
  \end{itemize}
\onslide<3->  \textbf{Exam:}
  \begin{itemize}[<+->]
    \item
      {\color{Mittel-Blau}$\SI{50}{\percent}$} of all points from
      the exercise sheets are needed
    \item content of exam: whole lecture \textbf{and all exercises} 
  \end{itemize}
\end{frame}

%-------------------------------------------------------------------------------

\begin{frame}{Organisation - Exercises 3 / 5}
  \begin{tabp}[1.0]
  \iitem{-2em}  \textbf{Exercises:}
    \item<2->
      Tutors: \LectureOrganisationTutors
    \item<3->
      Coordinators: \LectureOrganisationAdditional
    \item<4->
      Deadline: \LectureOrganisationExercisesWorkingTime
    \ssitem
  \item<5-> question please in Forum
  \citem \it exercises on Wedneday if needed
\item<6-> submission through ``commit'' through SVN and Daphne
\item<7-> unit test/checkstyle via  Jenkins
\item<8-> correction and marks  after one week
\citem \it{}  via ``update'' from SVN
\item<9-> achieved points can be seen in Daphne


  \end{tabp}
\end{frame}

%-------------------------------------------------------------------------------

\begin{frame}{Organisation - Exercises 4 / 5}
  \textbf{Exercises - Points:}
  \begin{itemize}
    \item
    Practical:
    \begin{itemize}
      \item
        {\color{Mittel-Blau}$\SI{60}{\percent}$} functionality
      \item
        {\color{Mittel-Blau}$\SI{20}{\percent}$} tests 
      \item
        {\color{Mittel-Blau}$\SI{20}{\percent}$} documentation, Checkstyle, etc.
      \item
        {\color{Greenb}Program is not running}
        $\Rightarrow$
        {\color{Greenb}0 points}
    \end{itemize}
    \item
    Theoretical (mathematical proof):
    \begin{itemize}
      \item
        {\color{Mittel-Blau}$\SI{40}{\percent}$} general idea / approach
      \item
        {\color{Mittel-Blau}$\SI{60}{\percent}$} clean / complete
    \end{itemize}
  \end{itemize}
\end{frame}

%-------------------------------------------------------------------------------

\begin{frame}{Organisation 5 / 5}
  \textbf{Effort:}
  \begin{itemize}
    \item
      {\color{Mittel-Blau}$\num{4}$ ECTS} (ESE),
%      {\color{Mittel-Blau}$\num{6}$ ECTS} (IEMS)
    \item
      {\color{Mittel-Blau}$\num{120}$} %/       {\color{Mittel-Blau}$\num{180}$}
      working hours per semester
    \item
      {\color{Mittel-Blau}$\num{14}$} Lectures each
      {\color{Mittel-Blau}$\SI{6}{\hour}$} %/
                                %{\color{Mittel-Blau}$\SI{8}{\hour}$} 
      + exam
    \item
      {\color{Mittel-Blau}$\SI{4}{\hour}$} %/
                                %{\color{Mittel-Blau}$\SI{6}{\hour}$}
 per exercise-sheet (one per week)
    \item Deadlines for IEMS more flexible 
  \end{itemize}
\end{frame}

\begin{frame}<beamer>{\LectureToC}
  \tableofcontents[currentsection, currentsubsection,
    subsubsectionstyle=show/show/shaded
  ]
\end{frame}

\subsubsection{Daphne}

\begin{frame}{Daphne}
  \textbf{Daphne:}
  \begin{itemize}
    \item
      Provides the following information:
      \begin{itemize}
        \item
          Name / contact information of your tutor
        \item
          Download of / info needed for exercise sheets
        \item
          Collected points of all exercise sheets
        \item
          Links to:
          \begin{enumerate}
            \item
              Coding standards
            \item
              Build system
            \item
              The other systems
          \end{enumerate}
      \end{itemize}
    \item
      Link: {\color{Mittel-Blau}\href{\LectureDaphneLink}{Daphne}}
  \end{itemize}
\end{frame}

%-------------------------------------------------------------------------------

\subsubsection{Forum}
\begin{frame}{Forum}
  \textbf{Forum:}
  \begin{itemize}
    \item
      Please don't hesitate to ask if something is unclear
    \item
      Ask in the forum and not separate.
      Others might also be interested in the answer
    \item
      One of the
      {\color{Mittel-Blau}tutors} will reply as fast as possible
    \item
      Link: {\color{Mittel-Blau}\href{\LectureForumLink}{Forum}}
  \end{itemize}
\end{frame}

%-------------------------------------------------------------------------------

\codeslide{python}{
\subsubsection{Checkstyle}
\begin{frame}{Checkstyle}{flake8}
  \textbf{Checkstyle (flake8):}
  \begin{itemize}
    \item
      Installation: \textbf{python} \texttt{-m pip install flake8}
    \item
      Check file: \textbf{flake8} \texttt{path/to/file.py}
    \item
      Check directory: \textbf{flake8} \texttt{path/to/directory}
    \item
      Link: {\color{Mittel-Blau}\href{\LectureCheckstyleLink}{flake8}}
  \end{itemize}
\end{frame}

%-------------------------------------------------------------------------------

%\subsubsection{Checkstyle}
%\begin{frame}{Checkstyle}
%  \textbf{Checkstyle (pep8):}
%  \begin{itemize}
%    \item
%      Installation: \textbf{pip} \texttt{install pep8}
%    \item
%      Update: \textbf{pip} \texttt{install --upgrade pep8}
%    \item
%      Check: \textbf{pep8} \texttt{--show-source --show-pep8 <file>}
%    \item
%      Link: {\color{Mittel-Blau}\href{\LectureCheckstyleLink}{pep8}}
%  \end{itemize}
%\end{frame}
}

%-------------------------------------------------------------------------------

\begin{frame}
  \frametitle{Unit Tests}
  \begin{tabl}
  \item Why Unit Tests?
  \eitem
  \begin{tabp}[0.9]
  \item \textbf{Reason 1:} A non-trivial method with out unit test is
    probably wrong
  \item \textbf{Reason 2:} simplifies debugging
  \item \textbf{Reason 3:} We and you can automatic check correctness of code
  \end{tabp}
\eitem
\item what is a good unit test?
\eitem
\begin{tabp}[0.9]
\item unit test checks desired output for given input
\item at least one \textbf{typical} input
\item at least one \textbf{critical} case, e.g. empty field in sorting etc.
\end{tabp}

  \end{tabl}
\end{frame}

%-------------------------------------------------------------------------------

\codeslide{python}{
\subsubsection{Test Example}
\begin{frame}{Test}{doctest}
  \textbf{Testing (doctest):}
  \vspace{-1.0em}
  \begin{columns}
    \begin{column}[t]{0.45\linewidth}
      \lstinputlisting[
        language=Python,
        style={python-idle-code},
        basicstyle=\footnotesize,
        tabsize=4,
        breaklines=false,
        emph={subOne},
        emphstyle=\color{blue}
      ]{Lecture/Code/DocTest.py}
    \end{column}
    \begin{column}[t]{0.6\linewidth}
      \begin{tabl}
        \item
          Run: \textbf{python} \textit{$<$file$>$}
        \item
          Debug: \textbf{python} \textit{$<$file$>$ -v}
        \eitem
      \item<2-> tests are in docstrings
      \item<3-> doctest runs them 
      \end{tabl}
    \end{column}
  \end{columns}
\end{frame}
}

%-------------------------------------------------------------------------------

%\subsubsection{Version management}
%\begin{frame}{Version management}{git}
%  \begin{itemize}
%    \item
%      Initialize directory: \textbf{git} \texttt{clone <URL> .}
%    \item
%      Add files/folders: \textbf{git} \texttt{add <file> --all}
%    \item
%      Create snapshot: \textbf{git} \texttt{commit -m "<Your Message>"}
%    \item
%      Upload: \textbf{git} \texttt{push origin master}
%    \item
%      Link: {\color{Mittel-Blau}\href{\LectureGitLink}{Git}}
%  \end{itemize}
%\end{frame}

%-------------------------------------------------------------------------------

\subsubsection{Version management}
\begin{frame}{Version management}{Subversion}
  \textbf{Version management (subversion):}
  \begin{itemize}
    \item
      Initialize / update directory: \textbf{svn} \texttt{checkout <URL>}
    \item
      Add files / folders: \textbf{svn} \texttt{add <file> --all}
    \item
      Create snapshot: \textbf{svn} \texttt{commit -m "<Your Message>"}
    \item
      Link: {\color{Mittel-Blau}\href{\LectureSubversionLink}{Subversion}}
  \end{itemize}
\end{frame}

%-------------------------------------------------------------------------------

\subsubsection{Jenkins}
\begin{frame}{Jenkins}
  \textbf{Jenkins:}
  \begin{itemize}
    \item is our build system
    \item you can check how the your  uploaded code compiles and runs
      \begin{itemize}
      \item especially whether all \textbf{unit tests} are passed
      \item and whether \textbf{Checkstyle (flake8)} is satisfied
      \end{itemize}
    \item will be shown tomorrow (exercise)
    \item
      Link: {\color{Mittel-Blau}\href{\LectureJenkinsLink}{Jenkins}}
  \end{itemize}
\end{frame}

\section{Sorting}

\begin{frame}{Sorting 1 / 2}
  \textbf{Problem:}
  \begin{itemize}
    \item
      Input: $n$ elements $x_1, \ldots, x_n$
    \item
      Transitive operator \enquote{\textbf{\color{Mittel-Blau}$<$}} which returns
      \textbf{\color{Mittel-Blau}true} if the left value is smaller than the 
      right one
      \begin{itemize}
        \item
          Transitivity: $x < y, \; y < z \rightarrow x < z$
      \end{itemize}
    \item
      Output: $x_1, \ldots, x_n$ sorted with operator
  \end{itemize}
  \onslide<2->
  \begin{exampleblock}{Example}
    Input:\hspace*{1.5em}14, 4, 32, 19, 8, 44, 65\\
    Output:
  \end{exampleblock}
  % \begin{itemize}[<3->]
  % \item Where do we need Sorting?
  %   \begin{itemize}
  %   \item in nearly every larger program
  %   \item \emph{example}: index construction for search engine.
  %   \end{itemize}
  % \end{itemize}
\end{frame}

%-------------------------------------------------------------------------------

\begin{frame}{Sorting 2 / 2}
  \textbf{Why do we need sorting?}
  \begin{itemize}
    \item
      Nearly {\color{Mittel-Blau}every} program needs a sorting-algorithm
    \item
      \textbf{Examples:}
      \begin{itemize}
        \item
          Index of a search engine
        \item
          Listing filesystem in explorer / finder
        \item
          (Music-) Library
        \item
          Highscore list
      \end{itemize}
  \end{itemize}
\end{frame}\subsection{MinSort}


\begin{frame}{MinSort - Algorithm}
  \begin{columns}
    \begin{column}{0.5\textwidth}
      \textbf{Informal description:}
      \begin{itemize}
        \item
          Find the minimum and switch the value with the
          {\color{Mittel-Blau}first} position
        \item
          Find the minimum and switch the value with the
          {\color{Mittel-Blau}second} position
        \item
          $\cdots$
      \end{itemize}
    \end{column}
    \begin{column}{0.5\textwidth}
      \begin{figure}[!h]%
        \begin{adjustbox}{width=\linewidth}
\rlap{\onslide<1>{\graw{1}{Rolfs-Images/minsort-o1.pdf}}}%
\rlap{\onslide<2>{\graw{1}{Rolfs-Images/minsort-o2.pdf}}}%
\rlap{\onslide<3>{\graw{1}{Rolfs-Images/minsort-o3.pdf}}}%
\rlap{\onslide<4>{\graw{1}{Rolfs-Images/minsort-o4.pdf}}}%
\rlap{\onslide<5>{\graw{1}{Rolfs-Images/minsort-o5.pdf}}}%
\rlap{\onslide<6>{\graw{1}{Rolfs-Images/minsort-o6.pdf}}}%
\rlap{\onslide<7>{\graw{1}{Rolfs-Images/minsort-o7.pdf}}}%
\rlap{\onslide<8>{\graw{1}{Rolfs-Images/minsort-o8.pdf}}}%
\rlap{\onslide<9>{\graw{1}{Rolfs-Images/minsort-o9.pdf}}}%
{\onslide<10->{\graw{1}{Rolfs-Images/minsort-o10.pdf}}}%
        \end{adjustbox}
        \caption{\textit{MinSort} at the fourth iteration onwards}%
        \label{fig:minsort_fourth_iteration}%
      \end{figure}%
    \end{column}
  \end{columns}
\end{frame}

%-------------------------------------------------------------------------------

\codeslide{python}{
\begin{frame}{MinSort - Algorithm}
  \textbf{MinSort in Python:}
  \lstinputlisting[
    language=Python,
    style={python-idle-code},
    basicstyle=\footnotesize,
    tabsize=4,
    emph={minsort},
    emphstyle=\color{blue}
  ]{Lecture/Code/MinSort.py}
\end{frame}
}

%TODO: Code C++, Java

%-------------------------------------------------------------------------------

\begin{frame}{MinSort - Runtime}
  \textbf{How long does our program run?}\vspace*{-0.5em}
  \begin{columns}%
    \begin{column}{0.415\textwidth}
      \begin{itemize}
        \item
          We test it for different input sizes
        \item
          \textbf{Observation:}\\
          It is going to be \enquote{disproportional}
          slower the more numbers are being sorted
      \end{itemize}
    \end{column}%
    \begin{column}{0.585\textwidth}%
      \vspace*{-1.0em}%
      \begin{table}[!h]%
        \caption{Runtime for \textit{MinSort}}%
        \label{tab:minsort_runtime}%
        \begin{tabular}{c|c}%
          $n$ & Runtime / \si{\milli\second}\\
          \midrule
          \num{2e3} & \num{5.24}\\
          \num{4e3} & \num{16.92}\\
          \num{6e3} & \num{39.11}\\
          \num{8e3} & \num{67.80}\\
          \num{10e3} & \num{105.50}\\
          \num{12e3} & \num{150.38}\\
          \num{14e3} & \num{204.00}\\
          \num{16e3} & \num{265.98}\\
          \num{18e3} & \num{334.94}
        \end{tabular}
      \end{table}
    \end{column}
  \end{columns}
\end{frame}

%-------------------------------------------------------------------------------

\begin{frame}{MinSort - Runtime}
  \textbf{How long does our program run?}\vspace*{-0.5em}
  \begin{columns}
    \begin{column}{0.415\textwidth}
      \begin{itemize}
        \item
          We test it for different input sizes
        \item
          \textbf{Observation:}\\
          It is going to be \enquote{disproportional}
          slower the more numbers are being sorted
      \end{itemize}
    \end{column}
    \begin{column}{0.585\textwidth}
      \begin{center}%
        \begin{figure}%
          \includegraphics[width=\textwidth]
            {Lecture/Images/MinSort/RuntimeSquared.png}%
          \vspace*{-1.0em}\caption{Runtime of \textit{MinSort}}%
          \label{fig:minsort_runtime}%
        \end{figure}%
      \end{center}
    \end{column}
  \end{columns}
\end{frame}

%-------------------------------------------------------------------------------

\begin{frame}{MinSort - Runtime}
  \begin{columns}%
    \begin{column}{0.5\textwidth}%
      \textbf{Runtime analysis:}
      \begin{itemize}
        \item
          In this lecture we study this diagram for \textit{MinSort}
          \begin{itemize}
            \item
              Thats what you should do in the first exercise sheet
          \end{itemize}
        \item
          \textbf{We observe:}\\
          \begin{itemize}
            \item
              The runtime {\color{Mittel-Blau}grows faster than linear}
            \item
              With double the input size we need four times the time
          \end{itemize}
      \end{itemize}
    \end{column}%
    \begin{column}{0.5\textwidth}%
      \begin{center}%
        \begin{figure}%
          \includegraphics[width=\textwidth]
            {Lecture/Images/MinSort/RuntimeSquared.png}%
          \vspace*{-1.0em}\caption{Runtime of \textit{MinSort}}%
          \label{fig:minsort_runtime_2}%
        \end{figure}%
      \end{center}
    \end{column}%
  \end{columns}
  \begin{columns}
    \begin{column}{1.0\textwidth}
      \begin{itemize}
        \item
          Next lecture we will analyze deeper with other methods
      \end{itemize}
    \end{column}
    % Dummy column
    \begin{column}{0.0\textwidth}\end{column}
  \end{columns}
\end{frame}\subsection{HeapSort}

\begin{frame}{HeapSort - Algorithm 1 / 10}
  \textbf{Heapsort:}
  \begin{itemize}
    \item
      The principle stays the same
    \item
      Better structure for finding the smallest element
  \end{itemize}
  \textbf{Binary heap:}
  \begin{itemize}
    \item
      Preferably a complete binary tree
    \item
      \textbf{Heap property:} Each child is smaller / larger than the parent
      element
  \end{itemize}
\end{frame}

%-------------------------------------------------------------------------------

\begin{frame}{HeapSort - Algorithm 2 / 10}
  \textbf{Min heap:}
  \begin{itemize}
    \item
      \textbf{Heap property:} Each child is {\color{Mittel-Blau}smaller}
      (larger) than the parent element
  \end{itemize}
\vspace*{-2em}
  \begin{columns}
    \begin{column}{0.45\textwidth}
      \begin{figure}[!h]
        \begin{adjustbox}{width=\linewidth}
          \input{Lecture/Images/Heap/MinHeap_Valid.tikz}
        \end{adjustbox}
        \caption{Valid min heap}
        \label{fig:minheap_valid}
      \end{figure}
    \end{column}%
    \hspace*{0.1em}%
    \begin{column}{0.45\textwidth}
      \begin{figure}[!h]
        \begin{adjustbox}{width=\linewidth}
\rlap{\input{Lecture/Images/Heap/MinHeap_Invalid_allblue.tikz}}%
\onslide<2->{\input{Lecture/Images/Heap/MinHeap_Invalid.tikz}}
        \end{adjustbox}
        \caption{Invalid min heap}
        \label{fig:minheap_invalid}
      \end{figure}
    \end{column}
  \end{columns}
\end{frame}

%-------------------------------------------------------------------------------

\begin{frame}{HeapSort - Algorithm 3 / 10}
  \textbf{How to save the heap?}\\[0.25em]
  \begin{itemize}
    \item
      We number all nodes from top to bottom and left to right starting at
      {\color{Mittel-Blau}0}
      \begin{itemize}
        \item
          The children of node {\color{Mittel-Blau}$i$} are
          {\color{Mittel-Blau}$2i + 1$} and {\color{Mittel-Blau}$2i + 2$}
%        \vspace*{0.5em}
        \item
          The parent node of {\color{Mittel-Blau}$i$} is
          {\color{Mittel-Blau}$\mathrm{floor}\left(\frac{i-1}{2}\right)$}
      \end{itemize}
  \end{itemize}%
\vspace*{-2.5em}%
\begin{columns}%
\begin{column}{0.5\textwidth}
      \begin{figure}[!h]%
        \begin{adjustbox}{width=\linewidth}
          \input{Lecture/Images/Heap/MinHeap_Numbered.tikz}%
        \end{adjustbox}
        \vspace*{-1.5em}%
        \caption{Min heap}%
        \label{fig:minheap_numbered}%
      \end{figure}%
    \end{column}
\onslide<9->
    \begin{column}{0.6\textwidth}
      \begin{table}[!h]
        \caption{Elements stored in array}
        \label{tab:minheap_numbered}
        \begin{tabular}{ccccccc}
          {\color{Greenb}0}&
          {\color{Greenb}1}&
          {\color{Greenb}2}&
          {\color{Greenb}3}&
          {\color{Greenb}4}&
          {\color{Greenb}5}&
          {\color{Greenb}6}\\
          \hline
          \multicolumn{1}{|c}{4}&%
          \multicolumn{1}{|c}{8}&%
          \multicolumn{1}{|c}{5}&%
          \multicolumn{1}{|c}{17}&%
          \multicolumn{1}{|c}{9}&%
          \multicolumn{1}{|c}{11}&%
          \multicolumn{1}{|c|}{7}\\
          \hline
        \end{tabular}\\
      \end{table}
       \ \ \ access to node $i$:\\
       \qquad{}  simply with \texttt{heap.get(i)}\\
       \qquad{} and \texttt{heap[i]}
    \end{column}
  \end{columns}
\end{frame}

%-------------------------------------------------------------------------------

\begin{frame}{HeapSort - Algorithm 4 / 10}
  \textbf{Taking the smallest element:}
  \begin{itemize}
    \item
      Remove he smallest element (root node)
    \item
      Replace the root with the last node
    \item
      {\color{Mittel-Blau}Sift} the root node until the
      {\color{Mittel-Blau}heap property} is statisfied
  \end{itemize}
  \begin{figure}[!h]%
    \begin{columns}%
      \begin{column}{0.3\textwidth}%
        \begin{adjustbox}{width=\linewidth}
\onslide<1>{\rlap{\renewcommand{\currentred}{Mittel-Blau}\input{Lecture/Images/HeapSort/MinHeap_Repair_First.tikz}}}%
\onslide<2->{\input{Lecture/Images/HeapSort/MinHeap_Repair_First.tikz}}%
        \end{adjustbox}%
      \end{column}%
      \hspace*{0.05em}%
      \begin{column}{0.3\textwidth}%
        \begin{adjustbox}{width=\linewidth}
\onslide<3>{\rlap{\renewcommand{\currentred}{Mittel-Blau}\input{Lecture/Images/HeapSort/MinHeap_Repair_Second.tikz}}}%
\onslide<4->{\input{Lecture/Images/HeapSort/MinHeap_Repair_Second.tikz}}%
        \end{adjustbox}
      \end{column}%
      \hspace*{0.05em}%
      \begin{column}{0.3\textwidth}%
        \begin{adjustbox}{width=\linewidth}
\onslide<5->{\input{Lecture/Images/HeapSort/MinHeap_Repair_Third.tikz}}%
        \end{adjustbox}%
      \end{column}%
    \end{columns}%
    \caption{Repair of a min heap}%
    \label{fig:minheap_repair}%
  \end{figure}
\end{frame}

%-------------------------------------------------------------------------------

\begin{frame}{HeapSort - Algorithm 5 / 10}
  \textbf{HeapSort:}
  \begin{itemize}
    \item
      Organize the {\color{Mittel-Blau}$n$} elements as heap
    \item
      While the heap contains more elements
      \begin{itemize}
        \item
          Take the smallest element
        \item
          Move the last node to the root
        \item
          Repair the heap like previously described
      \end{itemize}
    \item
      Output: {\color{Greenb}4}, \onslide<5->{\color{Greenb}5}
      \onslide<6->{,\color{Greenb}\ldots}
  \end{itemize}
  \vspace*{-0.5em}
  \begin{center}
   \begin{figure}[!h]%
      \begin{columns}%
        \begin{column}{0.3\textwidth}%
          \begin{centering}
            \begin{adjustbox}{height=8em}
              \hspace*{-2.5em} \input{Lecture/Images/HeapSort/HeapSort_First.tikz}%
            \end{adjustbox}%
          \end{centering}
        \end{column}%
        \begin{column}{0.3\textwidth}%
          \begin{centering}
            \begin{adjustbox}{height=8em}
       \hspace*{-0.75em}%
\rlap{\onslide<3>{\renewcommand{\currentred}{Mittel-Blau}\input{Lecture/Images/HeapSort/HeapSort_Second.tikz}}}%
\rlap{\onslide<4->{\input{Lecture/Images/HeapSort/HeapSort_Second.tikz}}}%
            \end{adjustbox}%
          \end{centering}
        \end{column}%
        \begin{column}{0.3\textwidth}%
          \begin{centering}
            \begin{adjustbox}{height=8em}
             \onslide<5->{\hspace*{-1em}\input{Lecture/Images/HeapSort/HeapSort_Third.tikz}}%
            \end{adjustbox}%
          \end{centering}
        \end{column}%
      \end{columns}%
      \caption{One iteration of HeapSort}%
      \label{fig:heapsort_repair}%
    \end{figure}
  \end{center}
\end{frame}

%-------------------------------------------------------------------------------

\begin{frame}{HeapSort - Algorithm 6 / 10}
  \textbf{Creation of a heap:}
  \begin{itemize}[<+->]
    \item
      This operation is called {\color{Mittel-Blau}heapify}
    \item
      The {\color{Mittel-Blau}$n$} elements are already in the containing array
    \item
      Interpret this field als binary heap with currently without the
      {\color{Mittel-Blau}heap property}
    \item
      We repair the heap from bottom up with {\color{Mittel-Blau}sift}
      \begin{center}
        \textit{bottom up in layers}
      \end{center}
  \end{itemize}
\end{frame}

%-------------------------------------------------------------------------------


\begin{frame}{HeapSort - Algorithm 7 / 10}
  \vspace{-1.0em}
  \begin{table}[!h]%
    \caption{Input in array}%
    \label{tab:heapify_numbers}%
    \begin{tabular}{ccccccc}
      {\color{Greenb}0}&
      {\color{Greenb}1}&
      {\color{Greenb}2}&
      {\color{Greenb}3}&
      {\color{Greenb}4}&
      {\color{Greenb}5}&
      {\color{Greenb}6}\\
      \hline
      \multicolumn{1}{|c}{11}&%
      \multicolumn{1}{|c}{7}&%
      \multicolumn{1}{|c}{8}&%
      \multicolumn{1}{|c}{3}&%
      \multicolumn{1}{|c}{2}&%
      \multicolumn{1}{|c}{5}&%
      \multicolumn{1}{|c|}{4}\\
      \hline
    \end{tabular}
  \end{table}
  \vspace*{-0.5em}
  \begin{centering}
    \begin{figure}[!h]%
      \begin{columns}%
        \begin{column}{0.425\textwidth}%
          \begin{adjustbox}{width=\linewidth}%
            \input{Lecture/Images/HeapSort/Heapify_First.tikz}%
          \end{adjustbox}%
        \end{column}%
        \begin{column}{0.425\textwidth}%
          \begin{adjustbox}{width=\linewidth}%
              \input{Lecture/Images/HeapSort/Heapify_Second.tikz}%
          \end{adjustbox}%
        \end{column}%
      \end{columns}%
      \caption{Heapify lower layer}%
      \label{fig:heapify_lower}%
    \end{figure}
  \end{centering}
\end{frame}

%-------------------------------------------------------------------------------

\begin{frame}{HeapSort - Algorithm 8 / 10}
  \begin{centering}
   \begin{figure}[!h]%
     \begin{columns}%
       \begin{column}{0.425\textwidth}%
         \begin{adjustbox}{width=\linewidth}%
           \input{Lecture/Images/HeapSort/Heapify_Third.tikz}%
         \end{adjustbox}%
       \end{column}%
       \begin{column}{0.425\textwidth}%
          \begin{adjustbox}{width=\linewidth}%
            \input{Lecture/Images/HeapSort/Heapify_Fourth.tikz}%
          \end{adjustbox}%
        \end{column}%
      \end{columns}%
      \caption{Heapify upper layer}%
      \label{fig:heapify_upper}%
    \end{figure}
  \end{centering}
\end{frame}

%-------------------------------------------------------------------------------

\begin{frame}{HeapSort - Algorithm 9 / 10}
  \begin{centering}
    \begin{figure}[!h]
      \begin{adjustbox}{width=0.425\linewidth}%
        \input{Lecture/Images/HeapSort/Heapify_Fifth.tikz}%
      \end{adjustbox}%
      \caption{Resulting heap}%
      \label{fig:heapify_upper_final}%
    \end{figure}
  \end{centering}
\end{frame}

%-------------------------------------------------------------------------------

\begin{frame}{HeapSort -  Runtime (intuitive)}
  \textbf{Finding the minimum:}
  \begin{itemize}
    \item
      \textbf{MinSort:} Iterate through all non-sorted elements
    \item
      \textbf{HeapSort:} finding minimum is trivial
      \begin{center}
\it{}        just take the root of the heap
      \end{center}
  \end{itemize}
  \vspace*{1.5em}
  \textbf{Removing the minimum in HeapSort:}
  \begin{itemize}
    \item
      Repair the heap and restore the {\color{Mittel-Blau}heap property}
      \begin{itemize}
        \item
          We don't have to repair the whole heap
      \end{itemize}
    \item
      More of this in the next lecture
  \end{itemize}
\end{frame}

\begin{frame}
  \frametitle{Literature / Links}
  \begin{tabl}
  \item For the Lecture in General:
  \eitem
  \begin{tabp}[0.9]
  \item Cormen / Leiserson / Rivest: Introduction to Algorithms
   \sitem\color{DarkGreen} [Classic course book. Table of content  available online, book has to be
  nought or borrowed.]
\sitem \color{DarkBlue}\href{http://mitpress.mit.edu/algorithms/}{\url{http://mitpress.mit.edu/algorithms/}}
\item Mehlhorn / Sanders: Algorithms and Data Structures, The Basic
  Toolbox
  \sitem\color{DarkGreen} [Newer textbook, more practical than
  Cormen/Leiserson/Rivest. Whole book online!]
\sitem \color{DarkBlue}\href{http://www.mpi-inf.mpg.de/~mehlhorn/Toolbox.html}{\url{http://www.mpi-inf.mpg.de/~mehlhorn/Toolbox.html}}
  \end{tabp}
\item Sorting:
  \begin{tabl}
  \item In Mehlhorn/Sanders:   5. Sorting and Selection
  \item In Cormen/Leiserson/Rivest:   II.7.1 HeapSort
  \item Wikipedia has also good entries for MinSort and HeapSort
  \end{tabl}
  \end{tabl}
\end{frame}

\end{document}


%%% Literature
\begin{frame}{\LectureFurtherLiterature}
  \begin{itemize}
    \item
    \textbf{General for this Lecture}
    \begin{btSect}{Literature/General/General}
      \btPrintAll
    \end{btSect}
  \end{itemize}
\end{frame}

%-------------------------------------------------------------------------------

\begin{frame}{\LectureFurtherLiterature}
  \begin{itemize}
    \item
      \textbf{Sorting}
      \begin{btSect}{Literature/eng/WikipediaSort}
        \btPrintAll
      \end{btSect}
  \end{itemize}
\end{frame}

%-------------------------------------------------------------------------------

\begin{frame}{\LectureFurtherLiterature}
  \begin{itemize}
    \item
      \textbf{Git}
      \begin{btSect}{Literature/General/Git}
        \btPrintAll
      \end{btSect}
  \end{itemize}
\end{frame}













%%% ===================================================================
%%% This should be at the END of the file !!!!!!
%%%
%%% Local Variables: ***
%%% mode: latex ***
%%% End: ***
%%% ===================================================================



